\chapter{Project management}\
\section{Project structure}

This project is a continuation of the previous research. When I first came into contact with the concept of combinatorial maps, it was very vague to me.  The only information I know about it is the basic structure of  the maps, so that I just implement a method to visualise the map with the fundamental steps. With the successful implementation of the first visualisation method, I learned more about the maps and drawing graph. According to prior knowledges accumulated in the previous research, I was quite clear about the content and purpose of the project. The essential purpose is still using different methods to visualise the structure, but a tool produced to enhance the project integrity and practicality. It also used to validate the result of methods. After discussing the ideas with supervisor and writing the proposal, the structure of project is more clearly. The first step is generating the frame of the tool. Filled the tool with different visualisation methods.

Validating whether each unit of the tool works well and  testing if results of the approaches perform well according to the specification writing in the proposal. All tests through the strict constrains. At the beginning, the visualiser page was separated into two pages, one for trying the specific examples of users and a randomly generation as the another page. Some fixed examples were used to test the methods, so that the early work is to creating the format page. For reducing the time to input the same string every time, a typical examples item was supported.Due to the same process for generating maps, the page of producing the random maps was merged into the same page.

Though the project has been finished successfully, there are still some problems can be improved. So many time is using to check the related knowledges that it is no sufficient time to implement every ideas who might be pursued later.

\section{Project log}

The project log was submitted weekly on the canvas, which also can be seem in the Appendix B. The log help me to summarise the knowledges learned and recording the problems appeared in the project.  It gives me enough time to consider what should do next week, as well.

\section{Supervisor meeting}

Meeting with the Project supervisor held once a week, to know what I have done in the previous week and discuss the problems I met. A clearer mind and explicit direction have been achieved after the discussion.

\section{Version control}
The GitLa is used to control the version of the project. The log of the Git helps us know what changes in each push operations. It will store the latest version of project after implementing a new function or fixing the problems of the project.