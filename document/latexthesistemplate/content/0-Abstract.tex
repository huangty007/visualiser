% !TeX encoding=utf8
% !TeX spellcheck = en-US

% \subsection*{Kurzfassung}

% <insert your name>
% \subsubsection*{<insert title>}


% %
% \mbox{}\\[0.5\baselineskip]\noindent
% \textbf{Schlagwörter:} 
% <insert key words>
% -----------------------------------------------------------------------
% \clearpage
% \subsection*{Abstract}

% <insert your name>
% \subsubsection*{<insert title (english)>}


% %
% \mbox{}\\[0.5\baselineskip]\noindent
% \textbf{Key words:} 
% <insert key words>

\setlength\parindent{0pt} 
% \setlength\parskip{\medskipamount}

% chapter without heading and without number
% \addchap*{Danksagung}
\addchap*{Abstract}
%
% Add your text here! You may take the following text as a guide:

The combinatorial map is a combinatorial topological structure. It represents a graph embedded into the surface. This project aims to search for different techniques for visualisation of combinatorial maps. The combinatorial map has many features, and the methods implemented in the project come from them. In order to attract more people learning the structure, the methods are collected as a tool in JAVASCRIPT. The methods and the visualiser are evaluated successfully though there are some problems. Some of the methods still can be improved in the future.
\\
\\
\textbf{Keywords:}Combinatorial maps, visualisation, topological structure

\par\needspace{1\baselineskip}%
\null\vfill

\parbox{\textwidth}{
\subsection*{Note}
Some content in section 3 has been based on the content written in my Mini-project report\cite{tingyu2019drawer}. 

\subsection*{Source code} 

The source code for this project can be found at https://git-teaching.cs.bham.ac.uk/mod-msc-proj-2018/txh755.
}