\chapter{Project log}
\label{chap:Appendix:B}
\textbf{week 1} This week mainly talk about the topics which were related to the combinatorial map. The first one is using different techniques to do the visualization of combinatorial maps and another one is to convert the pixels of images into combinatorial maps. For the first topic, there were two more methods can be used to visualize, using a spinning tree or using the third component faces of combinatorial maps.  As for the second topic, it needs more knowledge about it. All of these topics and approaches need me to do further research

The second task of the week was discussing which types of topics could I choose and how to perform the final work for the types. Two types were suitable to me, theory and software engineering, if I want to search the more methods to show the combinatorial map the first type might a good choice, however, if I can produce an interface to help the people know a lot about the combinatorial map, the second type is better. There is no doubt that this task is base on the first one. Hence, more information about the topics is needed to help make decisions and find the most appropriate project for me to do.

\textbf{week 2} The meeting this week had two parts, solving the problems in the proposal and identifying the details of the project.  After reading the proposal template, I found I was confusing about some items, for example, whether the project objective should be more specific or not. Therefore, the supervisor helped me to tackle the barrier. During this period, the approaches for dealing with the problem of the topic were settled, and we discussed if the ideas were feasible. Supervisor gave me more professional theoretical support and clearer research direction after listening to my description of the thoughts. 

\textbf{week 3} Discussing the details of the project through the proposal. First, a clear direction, since the project mainly cares about the approaches of visualization of combinatorial maps, it needs to involve all general combinatorial maps, rather than focus on the quite narrow type of maps like the maps based on the planar maps. As well,  certain the primary frame of the system via prototypes. 

\textbf{week 4} This week introduce the library and frame used to implement the designing of the sketch.

\textbf{week 5} Implement the sketch of the web page, and discuss the content of the report. The library for drawing the maps.

\textbf{week 6} The validating and parsing of the inputs and the supervise support some other idea for drawing the graph.

\textbf{week 7} Changing the structure of the pages, computing spanning tree and visualizing the permutation and cycles notations. Changing some terminology on pages.

\textbf{week 8} Implement drawing tree algorithm and visualizing tree-based combinatorial maps. Discussing faces-based algorithm and the methods to make the result clearer. Adding the useful introduction information to help freshman to understand the structure.

\textbf{week 9} Another tree-based method to avoid the cross problem. The problems occur in the project.

\textbf{week 10} Solving the problems occurred in the second tree-based method, and complete the method. Some suggestions about the layout of the page to collect all the type of inputs into one page.

\textbf{week 11} Finish the whole pages, fix the problem.

\textbf{week 12} Discussing details about the dissertation.

\textbf{week 13} Displaying the pages and discussing each section of the dissertation. Finishing the writing work.



