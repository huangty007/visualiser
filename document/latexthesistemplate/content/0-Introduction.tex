% !TeX encoding=utf8
% !TeX spellcheck = en-US

\chapter{Introduction}

\section{Motivation}

Combinatorial map is a combinatorial topological structure which describe a graph embedded in a surface. It includes three components vertices, faces, and edges, which are represent by the permutations of darts (half-edges). There are three notations of the permutations two-notation, one-line notation and cycle notation. All of the notations are used to represent the each component of the map, while, cycle notation and one-line are mentioned in the calculation frequently and two-line notation is utilised into visualisation generally. There are some strict constrains of a map, so that only if two maps are isomorphic by a homeomorphism of the underlying surfaces, they could be conjugation equivalent.

The drawing a topological structure is heated topic \cite{tamassia2013handbook}, while, the visualisation of combinatorial maps is rarely mentioned.This project aims to build a tool for visualising the combinatorial maps by using different techniques. Firstly, the tool could make the concept more accessible and interesting, which is quite important for the beginners who are new to the relevant knowledge. As the mean time, for those researchers who conduct related research, this tool could help them improve their efficiency and productivity effectively. 

Due to the final goal of the project is drawing the map automatically, the project will pay more attention on the layout of the map. The spanning tree and dual maps are discussed as the based sketch of the structure when visualising the maps.
%
% \begin{equation}
%   J_f(a) := \frac{\partial {f}}{\partial {x}}(a) 
%          := \frac{\partial(f_1,  \ldots, f_m)}{\partial(x_1, \ldots, x_n)}(a)
%          := \left(\frac{\partial f_i(a)}{\partial x_j}\right)_{i=1,\ldots,m;\
%              j=1,\ldots,n}
% \end{equation}

\section{Tool structure}

The Tool has two pages introduction and visualiser. The first page is to introduce the combinatorial maps and relevant knowledges involved in the visualiser.

The second page has three units. The head unit is also an introduction that offer the details of the page, namely, what the pages consist of, how to operate each unit and the representation of each terminology.
The input unit includes three types of input. In the first section some examples are supported for users to experience. For the second section, users can try or validate their own specific sample. Users can also play with the last section for generation a random map by only entering the number of darts. The last unit is the output part which contains the visualisation of each component of combinatorial maps with cycle and one-line notations, the other information of the maps and the final results of the methods. Each results can be clicked to view more details.

\section{Report structure}

The report structured as follow. Section 2 provides the related knowledge of combinatorial maps, as well as its purpose and significance. Section 3 describes the achievement in this area, the former method was proposed by others and the later method was what I have done in the last semester mini-project. Section 4 gives more specifications of the tool. The ideas of visualisation of maps and the construct of the tool are involved in section 5. Section 6 evaluates the methods and the system. Section7 details the achievements, issues and improvement of the system. Appendix A shows the file structure of the project and Appendix B records the weekly project log.