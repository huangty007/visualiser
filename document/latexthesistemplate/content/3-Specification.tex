% !TeX encoding=utf8
% !TeX spellcheck = en-US

\chapter{Specification}

\section{Calculating of combinatorial maps}

\subsection*{Parsing the user input which is limited to the cycle notation.}

\textbf{Correctness} The input can be parsed into the valid cycle notation of each permutation.

\textbf{Reliability} The tool can check the inputs at any time. If the input is illegal, there will be a space to show the warning information and the correct format of the input. All others operations must after checking and parsing the input.

\textbf{Performance} The correct result of parsing will display in the form of text and image.

\subsection*{Transforming cycle notation to one-line notation.}

\textbf{Correctness} These two notations can be converted to each other.

\textbf{Performance} The correct result will display in the form of text and image.

\subsection*{Computation of faces by given the permutations of vertices and edges. This particularly depends on the equation \(\phi=\sigma o\alpha\).}

\textbf{Correctness} Any of three component can be educes by others, i.e.,\(\phi\alpha\sigma=id\).

\textbf{Performance} The cycle notation and one-line of face will display in the form of text and image.

\subsection*{Generating a random combinatorial map by only given the number of darts (and vertices).}

\textbf{Correctness} The graph should be connected, and the composition of permutations must be identity.

\textbf{Performance} Showing each component with text and image.

\section{Visualising combinatorial maps}

\subsection*{The tool using different methods to visualise the combinatorial maps.}

\textbf{Correctness} The maps must be followed the constraints of the maps, that is, it must obey the orientation of each part. For the planar map, it must avoid the cross as much as possible. The whole structure must be connected and filled in the specific area without overflow.

\textbf{Clarity} User can have a clear mind when they look at the images. Each element must be discovered directly rather than distinguished for a long time.  Especially the edges should not be cramped together and have an explicit identity.

\textbf{Performance} Showing the results in the right area and have an effective introduction. The result must show in the limited time.

\textbf{Consistency} The result of the certain method for a map must be the same at any time. This requires the methods must be stable and robust. Otherwise, the results might confuse the users and they will misunderstand about the combinatorial maps.

\textbf{Aesthetics} The whole tool must have a clear structure and comfort colour matching. The layout of the visualisation must be reliable. A good interface and drawing will attract user exploring more about the combinatorial maps.

\section{Knowledge of combinatorial maps}

\subsection*{The tool aims to pass more information about combinatorial maps for whether they are beginners or experts in the relative area.  A introduction of the maps and tool is necessary.}

\textbf{Correctness} The background of the combinatorial maps must be correct which cannot be misleading.

\textbf{Performance} A introduction page to describe the tool and the structure. Other information like genus number can be shown in the visualisation part.
